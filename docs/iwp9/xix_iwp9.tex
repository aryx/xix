\documentclass{article}

\title{
XiX: Plan 9 in OCaml
}
\author{
Yoann Padioleau\\
yoann.padioleau@gmail.com
}

\begin{document}
\maketitle

\begin{abstract}
XiX is a work-in-progress port of parts of Plan 9 to OCaml.
\end{abstract}

\section{Introduction}

% XiX (which stands for Xix is Xix, first fully recursive), is a port of
% parts of plan9 to OCaml. Right now mk, rc, toolchain (assembler, linker
% C compiler) enough for helloprintf.c for arm, mips, riscv,
% but also utilities (grep? sed? awk? ed?), and windowing system rio!
% (with graphics!) and some rudimentary kernel.

% real original motiv was principia softwarica (see principia paper)
% to explain the code of plan9 (kernel, windowiong system, graphics and
% network stack, mk, rc, toolchain, etc.)
% Best way to fully understand is to port, to write the code yourself,
% even if port, and also test if needed all of it to fully understand
% when it does not work why need certain code.

% But now useful on its own.
% C is great but old. Rust, Ziglang new contenders.
% Actually some r9 projects for Rust, and ziglang kinda new toolchain)
% but OCaml has pros compared to Rust/Ziglang. For many programs, simpler
% to not handle memory. Actually original code was not doing much memory
% management. Simple mm. But even simpler now. And OCaml has lots
% of nice things like ADTs for more readable compilation in linker,
% no need optab.c, ocmp, ... just direct pattern matching on advanced
% SHOW code diff C vs OCaml.

% plan9 was rethinking and simplifying Unix. Xix is also rethinking
% and simplifying Plan9. Remove more features. Less code!
% even less code than plan9! STEPS project! So even easier 
% to teach. Actually series of books explaining the code, ocaml editions
% of C books in principia softwarica.

%TOC:
% - components:
%   * omk/orc, less features, just essential, so LOC/LOE comparison?
%     can even do mk+rc in same binary? rc as a lib! easy with OCaml!
%     so even better than plan9! perf improv?
%   * toolchain: assembler/linker/compiler, ar, more reusable code!
%     reuse more in asm, linker, compiler! lots of code dupe in plan9
%     use 'a instr polymorphic type, helpers, arch interfaces, etc.
%     nice use of marshalling to reduce code of IO and special object/ar formats
%     archs: arm, mips, riscv (RISC is simpler)
%     OS: plan9 and Linux ELF
%   * orio and threads, and 9p in OCaml for draw device interaction
%     significant! can replace rio by orio on plan9 (more on OCaml for plan9 below)
%     screenshot?
%   * oed, and other utilities: oawk, ogrep, osed
%   * olex, oyacc
%   * git! not in plan9
%   * lib_core, lib_graphics
%   * kernel (rudimentary)
% - LOC/LOE! compare books on mk vs omk
% - on OCaml
%   also capabilities!
% - git, docker, GHA, testsuite (not much tests in plan9), bootstrap-mk.sh,
%   easy to test on Linux/macOS/Windows (plan9 not so easy by comparison,
%   might have slow down also adoption)
% - ocaml for plan9 (and myocaml)
% - Software engineering lessons and improvements after 30 years:
%   abuse int, abuse strings, code dupe, error mangagement OK_0, OK_1,
%   bools, ...
%   we learned stuff in 30 years.
% - Future work

\section{Conclusion}

\end{document}
